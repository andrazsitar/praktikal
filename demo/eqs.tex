\documentclass[12pt,a4paper]{article}
\usepackage[slovene]{babel}
\usepackage{xcolor} \pagecolor[HTML]{1E1E1E} \color[HTML]{FFFFFF}
\usepackage[hidelinks]{hyperref}
\usepackage{graphicx}
\usepackage{amsmath}
\usepackage{amssymb}
\usepackage{physics}
\usepackage{float}

\pdfinfo{
  /Title (Fizikalni praktikum; Semester N; Arhetip)
  /Author (Andraž Sitar)}

\hyphenpenalty=50000
\newcommand\pkt[1]{\underline{\text{\$ praktikal\_{#1} \$ }}}
\renewcommand{\vec}{\vb}
\renewcommand{\phi}{\varphi}
\renewcommand{\epsilon}{\varepsilon}

\newcommand{\qmarks}[1]{``{#1}''}

\title{Arhetip}
\author{Andraž Sitar}
\date{}

\begin{document}
\maketitle

\tableofcontents
\pagebreak

\section{Uvod}
Teorija ter izpeljave enačb.
\pagebreak
\section{Naloge}
Opombe in razjasnitve pri meritvah. Začetek in konec vnosa in obdelave podatkov označimo z $\backslash$pkt\{eqsb\} in $\backslash$pkt\{eqse\}

Imena spremenljivk so na levi (vključno z enotami), njihova definicija pa na desni strani enačaja. Uvoženih spremenljivk privzeto ni v končni datoteki.
\pkt{eqsb}
\begin{eqnarray*}
	l_{primer}[m]	&=& (100 \pm 1) \cdot 10^{-9} \\
	a[m]	&=& (\$tab1[a] \pm 0.1) \cdot 10^{-3} \\
	r_{1}	&=& \$tab2[r_{1}] \pm 0.001 \\
	r_{2}	&=& \$tab2[r_{2}] \pm 0.001 \\
	r_{3}	&=& \$tab2[r_{3}] \pm 0.001 \\
	a_{1}	&=& 2 \cdot a \\
	b[s]	&=& \$tab1[b] \pm 0.5 \\
	N	&=& \mathbb{N}^{12} \cdot 10^{3} \\
	M	&=& \$tab1[M] \\
\end{eqnarray*}
\pkt{eqse}
Kot vrednost opazimo vektor, ki predstavlja zaporedne meritve, število ob tropičju pa predstavlja število meritev, ki niso prikazane.
\\

Uvožene spremenljivke pa lahko vidimo v tej tabeli.
\pkt{tab}{$ l_{primer} \& a \& b \& a_{sest} \& N \& M $}

\pagebreak
Bolj nazorno je meritve ter izračune prikazati na grafu.
\pkt{fig}{$a \{ a \& a_{1} \} / b$; error=true; leg=true; fit=lin; title=-||-,ter,dodaten,opis}
Grafu odvisnosti spremenljivk lahko spreminjamo nastavitve, nekatere izmed njih so že uporabljene. 

\pagebreak
Lahko narišemo tudi histogram.
\pkt{fig}{$\{r_{1} \& r_{2} \& r_{3} \} \sim ?$; nBins=20; dens=true; leg=true; type=bar; stacked=true; title=-||-,ter,dodaten,opis}


\pagebreak
Sledijo enačbe, izpeljane v uvodu.
\pkt{eqsb}
\begin{eqnarray*}
	a_{sest}	&=& a \oplus a_{1} \\
	K_{izm}	&=& \derivative{(a)}{b} \\
	\\
	r_{1,avg}	&=& \overline{r_{1}} \\
	s_{1}	&=& \sigma_{r_{1}} \\
	r_{2,avg}	&=& \overline{r_{2}} \\
	s_{2}	&=& \sigma_{r_{2}} \\
	r_{3,avg}	&=& \overline{r_{3}} \\
	s_{3}	&=& \sigma_{r_{3}} \\
\end{eqnarray*}
\pkt{eqse}
Vidimo, da so pri izračunih enote količin ustrezne. Program preveri različne kombinacije enot ter sešteje potence posameznih enot v ulomku (ali produktu) ter izbere tisto, katere vsota potenc je najmanjša. Zato morda izbere neustrezen, krajši zapis enote. Sestavljene enote lahko definiramo v \qmarks{configu}.

\pagebreak
\section{Povzetek}
Kratek povzetek poteka meritev, primerjava rezultatov s kolegi po Svetu, npr. $K_{izm} = \pkt{ref}{K_{izm}}$, kar vidimo na sliki (\ref{fig:plot_a;b}) (ime oznake najdemo v izhodnem dokumentu), ter komentar rezultatov.

\end{document}
